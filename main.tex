% Actualizacion 2024: Logotipos de la UPV


\documentclass[12pt]{article}
\usepackage[left=2.5cm,top=2.0cm,right=2.5cm,bottom=3.0cm]{geometry}
\usepackage[utf8]{inputenc}
\usepackage[spanish]{babel}
%\usepackage[english, american]{babel}
%\usepackage[spanish,es-tabla]{babel}
\usepackage[linguistics]{forest}
\usepackage{amssymb, amsmath, amsbsy} % simbolitos
\usepackage{longtable} % para tablas largas
\usepackage{graphicx}
\usepackage{fancyhdr}
\usepackage{xcolor}
\usepackage{multirow}
\usepackage{listings}
\usepackage{caption}
\usepackage{subcaption}
%\usepackage{parskip}
\usepackage[skip=12pt plus1pt]{parskip}
\usepackage{pdfpages} % Incluir PDF en documento en LATEX
\usepackage{verbatim} % comentarios
\usepackage{algpseudocode}
\usepackage{algorithm}
\usepackage{pdflscape}
\usepackage{multirow}
\usepackage{afterpage}
\usepackage{array,booktabs,ragged2e}


\usepackage[T1]{fontenc}      % idioma español
\usepackage{amsmath}          % símbolos matemáticos
\usepackage{csquotes}         % para \enquote{}



%\newcolumntype{R}[1]{>{\RaggedLeft\arraybackslash}p{#1}}
\newcolumntype{L}[1]{>{\raggedright\let\newline\\\arraybackslash\hspace{0pt}}m{#1}}
\newcolumntype{C}[1]{>{\centering\let\newline\\\arraybackslash\hspace{0pt}}m{#1}}
\newcolumntype{R}[1]{>{\raggedleft\let\newline\\\arraybackslash\hspace{0pt}}m{#1}}

\floatname{algorithm}{Algoritmo}
\renewcommand{\listalgorithmname}{Lista de algoritmos}
\renewcommand{\algorithmicrequire}{\textbf{Entrada:}}
\renewcommand{\algorithmicensure}{\textbf{Salida:}}






% Comentario con respecto a las referencias:
% Por defecto este documento utiliza el formato IEEEtr para formatear las referencias. Si algun asesor requiere formato APA, solo comente la siguiente linea y descomente la linea debajo. Tambien para que las citas funcionen de manera adecuada, el paquete label debe tener las opciones mostradas, o en su defecto, el año en el listado no lo muestra correctamente \usepackage[english, american]{babel}

% Para utilizar el formato de citas IEEE y comentar los dos parrafos siguientes
\usepackage[backend=bibtex,sorting=none]{biblatex}
% Para utilizar el formato APA, sugiero comentar la linea anterior y descomentar las dos proximas lineas
%\usepackage[backend=biber,style=apa]{biblatex}
%\DeclareLanguageMapping{english}{american-apa}

\makeatletter
\DefineBibliographyExtras{spanish}{%
  \setcounter{smartand}{1}%
  \let\lbx@finalnamedelim=\lbx@es@smartand
  \let\lbx@finallistdelim=\lbx@es@smartand
}
\renewbibmacro*{name:delim:apa:family-given}[1]{%
  \ifnumgreater{\value{listcount}}{\value{liststart}}
    {\ifboolexpr{
       test {\ifnumless{\value{listcount}}{\value{liststop}}}
       or
       test \ifmorenames
     }
       {\printdelim{multinamedelim}}
       {\lbx@finalnamedelim{#1}}}
    {}}
\makeatother


% Estas lineas permiten romper los hipervinculos muy largos en las referencias!!!!
\setcounter{biburllcpenalty}{7000}
\setcounter{biburlucpenalty}{8000}
\addbibresource{x.bib} % ARCHIVO DE BIBLIOGRAFÍA


%\usepackage{url}
\usepackage[bookmarks=true,breaklinks=true,bookmarksopen=false,colorlinks=true,linkcolor=blue]{hyperref}
\usepackage[hyphenbreaks]{breakurl}
% Regla que define explicitamente que caracteres rompen los hipervinculos para separar las lineas
%https://es.overleaf.com/11089898rhgykrqyqytx
% Actualiza en automático la fecha de las citas de internet a la fecha de la compilación del documento
\usepackage{datetime}
\newdateformat{specialdate}{\twodigit{\THEDAY}-\twodigit{\THEMONTH}-\THEYEAR}
%\newdateformat{specialdate}{\twodigit{\THEDAY}-\THEYEAR}
\date{\specialdate\today}

\newcommand{\HRule}{\rule{\linewidth}{0.25mm}}


% CONSTANTES NECESARIAS PARA EL DOCUMENTO ---> MODIFIQUEN A SU CRITERIO
\newcommand{\ncarrera}            {Ingeniería en Tecnologías de la Información}
\newcommand{\nasesorinstitucional}{Dr. Marco Aurelio Nuño Maganda}
\newcommand{\NombreAlumno}{Jose Manuel Alonso Cepeda}
%Hombres cambien LA por EL
\newcommand{\elolaNombreAlumno}{el}  
\newcommand{\OA}           {o}  %Hombres cambien A por O
\newcommand{\Matricula}           {1730505}  %SU MATRICULA
\newcommand{\NombreProyecto}{Diseño e implementación de un middleware para la interoperabilidad entre la plataforma Nez y el concentrador de servicios Jub}
\newcommand{\fechacarta}{26 de Abril de 2025}
\newcommand{\ncuatrimestre}{Mayo-agosto 2025}
\newcommand{\nevalador}{Dr. Hiram Herrera Rivas}
\newcommand{\FechaExposicion}{11 de Agosto de 2021}
\newcommand{\HoraExposionFormatoVenticuatroHoras}{10:00}
\newcommand{\elolaNombreEmpresa}{la}  %Si la empresa es femenino (por ejemplo universidad, usen la) o masculino (el instituto) pongan el
\newcommand{\organismoreceptor}   {CENTRO DE INVESTIGACIÓN Y DE ESTUDIOS
 AVANZADOS DEL INSTITUTO POLITÉCNICO NACIONAL (CINVESTAV) UNIDAD TAMAULIPAS}

% NOTA: Dos diagonales juntas (\\) indican un saldo de linea. En este caso particular hay 2 (el titulo se ajusta a tres lineas, porque es muy largo. Hacer las adecuaciones pertinentes
\newcommand{\NombreProyectoheader}     {Notificación de actualizaciones de la información \\en un sistema de información web de pacientes\\ hospitalizados a sus familiares del Sagrado Corazón de Jesus  }
\newcommand{\nasesorempresaria}   {DR. JOSÉ LUIS GONZÁLEZ COMPEÁN}
\newcommand{\fechaPortada}               {Junio de 2021}





\newcommand{\separacionCorta}{0.0cm}
\newcommand{\separacionLarga}{0.5cm}

\usepackage[overload]{textcase}
\newcommand{\iemph}[1]{\MakeTextUppercase{#1}}

\pagestyle{fancy}
\headheight 45pt
\fancyhead{} % Clear all header fields
\fancyhead[L]{\includegraphics[height=1.00cm]{UTYP.png}}%
%\fancyhead[C]{\begin{center}\NombreProyectoheader\end{center}}%
\fancyhead[C]{\begin{center}\scriptsize{\NombreProyectoheader}\end{center}}%
%\fancyhead[R]{\includegraphics[height=1.5cm]{LogoUPV_2019.png}}%
\fancyhead[R]{\includegraphics[height=1.25cm]{LogoUPV_2023.png}}%
\fancyfoot[R]{\thepage} % Clear all footer fields 
\fancyfoot[C]{}
\fancyfoot[L]{}

\DefineBibliographyStrings{english}{%
  references = {Referencias},% replace "references" with "bibliography"  for `book`/`report`
}

\addto\captionsenglish{%
  \renewcommand{\figurename}{Figura}%
  \renewcommand{\tablename}{Tabla}%
} 

\usepackage{wallpaper}
 
 
%\renewcommand{\figurename}{Figura}
%\renewcommand{\tablename}{Tabla}

 
\begin{document}

%-----------------------------------------------------------------------------------------------------------------
% PAGINA 1 - PORTADA
\setcounter{page}{1}
\pagenumbering{roman}
\thispagestyle{empty}

\begin{center}

\begin{tabular}{cp{5cm}c}
\includegraphics[height=2.25cm]{UTYP.png} & 
& \includegraphics[height=2.25cm]{LogoUPV_2023.png}   \\
\end{tabular}

\Large \textbf{UNIVERSIDAD POLITÉCNICA DE VICTORIA}
\vspace{0.5cm}
\hrule
\vspace{0.1cm} 
\hrule
\vspace{0.5cm}


%\HRule \\[\separacionCorta]
\textbf{\iemph{\NombreProyecto}} \\[\separacionLarga]
%\Large \textbf{TESINA}
%\HRule \\[\separacionLarga]
T E S I N A \\
QUE PARA OBTENER EL GRADO DE \\
\textbf{\iemph{\ncarrera}} \\[\separacionLarga]

PRESENTA: \\[\separacionCorta]
%\textbf{\Capitalize{\NombreAlumno}\\[\separacionLarga]
\textbf{\iemph{\NombreAlumno}}\\[\separacionLarga]
%EN CUMPLIMIENTO DE \\[\separacionCorta]
%LA ESTADÍA DE LA CARRERA DE \\[\separacionCorta]


DIRECTOR \\[\separacionCorta]
\textbf{\iemph{\nasesorinstitucional}} \\[\separacionCorta]

CO-DIRECTOR \\[\separacionCorta]
\textbf{\iemph{\nasesorempresaria}} \\[\separacionCorta]

ORGANISMO RECEPTOR \\[\separacionCorta]
\textbf{\iemph{\organismoreceptor}} \\[\separacionLarga]

\end{center}
\begin{flushright}
\iemph{Ciudad Victoria, Tamaulipas, \fechaPortada}
\end{flushright}

\HRule 



% En las siguientes 3 paginas, debe incluir una digitalización de calidad de sus respectivas cartaas, no una fotografia toda cucha tomada con su celular de Coppel
% Pagina 1: Digitalización de la carta de presentación (sustituir por una original en el empastado)
\clearpage
\thispagestyle{fancy}
\pagenumbering{roman}
\setcounter{page}{1}
\includepdf[pages={1}]{CartaPresentacion.pdf}

% Pagina 2: Digitalización de la carta de aceptación (sustituir por una original en el empastado)
\clearpage
\thispagestyle{fancy}
\includepdf[pages={1}]{CartaAceptacion.pdf}

% Pagina 3: Digitalización de la carta de liberación (sustituir por una original en el empastado)
\clearpage
\thispagestyle{fancy}
\includepdf[pages={1}]{CartaLiberacion.pdf}

% Pagina 4: Carta de aceptación del ASESOR INSTITUCIONAL(sustituir por una original en el empastado)
\clearpage
\ULCornerWallPaper{1}{Membrete_2023_Sin_Colores_Diabolicos.pdf}
\thispagestyle{empty}

\vspace*{1.5cm}
\large

\begin{center}
\textbf{CARTA DE ACEPTACIÓN DEL DOCUMENTO PARA SU IMPRESIÓN}\\[\separacionLarga]
\end{center}

\begin{flushright}
Cd. Victoria, Tamaulipas a \fechacarta \\[\separacionLarga]
\end{flushright}

\parindent=0mm

\NombreAlumno \\
PRESENTE \\[\separacionCorta]

Le comunico que  el Programa Académico de \ncarrera\ \ le ha otorgado la autorización para la impresión de su Tesina de Estadía Práctica cuyo título es: \\[\separacionCorta]

\begin{center}
\textbf{\NombreProyecto} \\[\separacionLarga]
\end{center}

\begin{center}	
\begin{tabular}{ccc}
\centering
& ATENTAMENTE & \\ 
& & \\
& & \\
& & \\ \hline
& \nasesorinstitucional & \\
& ASESOR INSTITUCIONAL & \\
\end{tabular}
\end{center} 
\vspace{2cm}
c.c.p Director de programa académico


%-----------------------------------------------------------------------------------------------------------------
\clearpage
\ClearWallPaper
\thispagestyle{empty}
\newgeometry{left=1.5cm,top=1.0cm,right=1.5cm,bottom=2.0cm}             
\begin{landscape}

%\afterpage{\restoregeometry}


\begin{tabular}{p{3cm}p{16cm}p{3cm}}
\multirow{4}{*}{\includegraphics[width=3.0cm]{UTYP.png}} &  & \multirow{4}{*}{\includegraphics[width=3.0cm]{LogoUPV_2023.png}} \\
   & \multicolumn{1}{c}{\textbf{EVALUACIÓN DE ESTADÍA}} &  \\ %\cline{2-2}
& \multicolumn{1}{c}{\textbf{Rúbrica para evaluación de la presentación y el reporte de estadía}} & \\ 
& & \\ 
\end{tabular}

\normalsize
\begin{tabular}{p{13cm}R{10cm}}
\multicolumn{1}{l}{Nombre del alumno: \underline{\textbf{\iemph{\NombreAlumno}}}}  & 
Calificación final: \underline{\hspace{3cm}}    \\ %\cline{2-2}
\multicolumn{2}{c}{Periodo: \underline{\textbf{\iemph{\ncuatrimestre}}}}    \\ %\cline{2-2}
& \\
\end{tabular}

\scriptsize 
%\begin{center}
%\resizebox{\linewidth}{!}{%
\begin{tabular}{C{2.0cm}|C{2.0cm}|C{4.5cm}|C{4.5cm}|C{4.5cm}|C{4.5cm}}
\hline
\multirow{2}{*}{Ponderación} & Aspecto a  & Competente & Independiente & Básico Avanzado & No Competente \\
 & Evaluar & 10 & 9 & 8 & 5 \\ 
\hline
40 & Resultados y Actividades & Estrechamente relacionados al perfil de
egreso de su programa académico & Parcialmente relacionados al perfil de
egreso de su programa académico & Escasamente relacionados al perfil de egreso de su programa académico & Escasamente relacionados al perfil de egreso de su programa académico \\  \hline

30 & Exposición de las actividades de la estadía  & Detalladas y sustentadas con respecto a los resultados que se obtuvieron & Detalladas y sustentadas parcialmente con respecto a los resultados que se obtuvieron & Detalladas parcialmente con respecto a los resultados que se obtuvieron & Detalladas escazamente con respecto a los resultados que se obtuvieron \\  \hline 

10 & Material visual Lenguaje
verbal & Uso el lenguaje y la terminología
apropiadas; El material visual está organizado,
adecuado y suficiente
 & 
Uso el lenguaje y la terminología apropiadas
El material visual está parcialmente
organizado y es suficiente
& 
Uso el lenguaje y la terminología son parcialmente apropiadas; El material visual está parcialmente
organizado y es suficiente
 & 
Uso el lenguaje y terminología es inapropiado; El material visual no está organizado y es insuficiente
 \\  \hline
10 & Exposición en Idioma Inglés &

Pronunciation is clear so language is easily understood (2.5) Uses fluent connected speech, occasionally disrupted by search for correct form of expression (2.5) Uses topic related vocabulary without problems (2.5) Responds to questions using varied and descriptive vocabulary and language structures (2.5) & 

Pronunciation is understandable, but there are slight errors (2.25)
Speech is connected but frequently disrupted by search for correct form of
expression (2.25) Uses some topic related vocabulary sufficient to communicate ideas (2.25) Responds to questions using simple but accurate vocabulary and language structures (2.25 & 

Pronunciation is understandable most of the time, marked native accent and many errors
(2) Speaks with simple sentences, sometimes not connected, but is understood (2) Uses basic vocabulary to communicate ideas
(2) Partly responds to simple questions, with limited vocabulary and language structures (2)  &

Pronunciation makes language very difficult to understand (1) Uses one-word/two-word utterances (1) Unable to communicate ideas due to lack of vocabulary (1) Uses isolated words or sentence fragments to respond to questions (1) 
\\  \hline

5 & Respuesta a los cuestionamientos de los evaluadores & 
Clara y satisfactoria & 
Clara y parcialmente satisfactoria & 
Clara e insuficiente & 
Confusa e insuficiente 
\\  \hline
5 & Autorización de tesina en tiempo y forma & 
Presenta en tiempo y forma 
\tikz[overlay, remember picture,anchor=base]  \node (Mark){}; 
 & 
Presenta en tiempo y forma con la mayoría de requerimientos solicitados & 
Presenta en tiempo y con algunas limitantes  de los requerimientos solicitados. &
Presenta fuera de tiempo y con los mínimos requerimientos solicitados.
 \\  % Una linea en Blanco para poner la marca del ASESOR INSTITUCIONAL
&
&
%\tikz[overlay, remember picture,anchor=base] \node (Center){};

&
 \\  \hline

\end{tabular}

%\begin{tikzpicture}[remember picture, overlay, note/.style={rectangle callout, fill=#1}]
%\node [note=red!80, callout absolute pointer={(Mark)}] at (Center) {COMPETENTE!};
%\end{tikzpicture}

%}
%\end{center}

\normalsize


%\begin{center}
%\begin{tabular}{ccc}
%\includegraphics[scale=0.10]{FirmaMANM2.png} & & \\
%\hline 
%\nasesorinstitucional & \nevalador & \\
%ASESOR INSTITUCIONAL & EVALUADOR & EVALUADOR INGLÉS \\
%\end{tabular}
%\end{center}

\begin{center}
\begin{tabular}{cp{1cm}cp{1cm}c}
%\includegraphics[scale=0.10]{ArchivoFirmaASESOR.png} & & & & \\
 & & & & \\
  & & & & \\
\hline 
\nasesorinstitucional & & \nevalador & & \\
ASESOR INSTITUCIONAL & &EVALUADOR & &EVALUADOR DE INGLÉS \\
\end{tabular}
\end{center}


\end{landscape} 
\restoregeometry

%-----------------------------------------------------------------------------------------------------------------
% Pagina 5: Registro de Evaluación de Estadía
\clearpage
\pagestyle{empty}
\ULCornerWallPaper{1}{Membrete_2023_Sin_Colores_Diabolicos.pdf}

\vspace*{0.5cm}



\large
\begin{center}
\textbf{REGISTRO DE EVALUACIÓN DE EXPOSICIÓN DE ESTADÍA}
\\[\separacionLarga]
\end{center}


Siendo las \HoraExposionFormatoVenticuatroHoras \ \ horas del día \FechaExposicion, \elolaNombreAlumno\ \ alumn\OA\ \ \textbf{\NombreAlumno}, del programa académico \textbf{\ncarrera}, con matricula \textbf{\Matricula}, presentó la exposición de la estadía realizada durante el cuatrimestre \textbf{\ncuatrimestre}, en \elolaNombreEmpresa\ \ \textbf{\organismoreceptor}, con el proyecto titulado \textbf{\NombreProyecto}.\\

Una vez concluido el proceso de evaluación, y con base a la rúbrica establecida para éste propósito, se determina que la calificación de la estadía es \underline{\hspace{2cm}}. %\hrulefill.

\begin{center}
\begin{tabular}{ccc}
& & \\
& & \\
& & \\
%& \includegraphics[scale=0.10]{FirmaAsesorPro.png} & \\
\hline 
%& \underline{\hspace{8cm}}& \\
& \nasesorinstitucional & \\
& ASESOR INSTITUCIONAL & \\
& & \\
& & \\
& & \\
%& \underline{\hspace{8cm}}& \\
\hline 
& \nevalador & \\
& EVALUADOR & \\
& & \\
& & \\
%& & \\
\hline 
& & \\
& EVALUADOR DE INGLÉS & \\
\end{tabular}
\end{center}

\normalsize

%-----------------------------------------------------------------------------------------------------------------
% PAGINA 4 - AGRADECIMIENTOS
\clearpage
\ClearWallPaper
\pagestyle{fancy}
\section*{\centering Agradecimientos}
\addcontentsline{toc}{section}{Agradecimientos}
%\input{Agradecimientos.tex}

%-----------------------------------------------------------------------------------------------------------------
% PAGINA 5 - RESUMEN EN ESPAÑOL

\clearpage
\section*{\centering Resumen}
\addcontentsline{toc}{section}{Resumen}
%\input{Resumen.tex}
El avance de la ciencia de datos y el aprendizaje profundo depende críticamente de la capacidad para integrar sistemas de software heterogéneos y especializados. Plataformas como \textit{Nez}, un \textit{framework} para el procesamiento de datos a gran escala basado en el modelo \textit{PuzzleMesh}, y \textit{Jub}, un concentrador de servicios para el monitoreo de fenómenos atmosféricos, ofrecen capacidades potentes pero operan en silos aislados, limitando su potencial sinérgico. Este trabajo aborda el desafío de la interoperabilidad entre estos dos sistemas mediante el diseño e implementación de un \textit{middleware} de acoplamiento ligero. 

La solución propuesta se basa en una arquitectura de API \textit{RESTful}, que actúa como un puente de comunicación estandarizado, permitiendo que \textit{Jub} invoque los servicios de procesamiento avanzado de \textit{Nez} de manera transparente y eficiente. El \textit{middleware} implementado no solo facilita el intercambio de datos y procesos en tiempo real, sino que también establece las bases para la creación de una malla de servicios de ciencia de datos unificada, gestionando las operaciones de almacenamiento a través del cliente \textit{MictlanX}. 

El prototipo fue desarrollado utilizando el \textit{framework} \textit{FastAPI} de Python, seleccionado por su alto rendimiento y sus capacidades para la rápida creación de APIs robustas. La validación del sistema se realizó a través de un caso de uso de procesamiento de imágenes, demostrando la viabilidad y eficacia del \textit{middleware} como catalizador para la integración de sistemas complejos en entornos de investigación científica.  

\textbf{Palabras clave:} Middleware, Interoperabilidad, Malla de Servicios, API REST, FastAPI, Nez, Jub, Sistemas Distribuidos.


%-----------------------------------------------------------------------------------------------------------------
% PAGINA 6 - RESUMEN EN INGLES

\clearpage
\section*{\centering Summary}
\addcontentsline{toc}{section}{Summary}
%\input{Summary.tex}
The advancement of data science and deep learning critically depends on the ability to integrate heterogeneous and specialized software systems. Platforms such as \textit{Nez}, a framework for large-scale data processing based on the \textit{PuzzleMesh} model, and \textit{Jub}, a service hub for monitoring atmospheric phenomena, offer powerful capabilities but operate in isolated silos, limiting their synergistic potential. This work addresses the challenge of interoperability between these two systems through the design and implementation of a lightweight \textit{middleware}. 

The proposed solution is based on a RESTful API architecture, which acts as a standardized communication bridge, allowing \textit{Jub} to invoke \textit{Nez}'s advanced processing services transparently and efficiently. The implemented \textit{middleware} not only facilitates the real-time exchange of data and processes but also lays the foundation for creating a unified data science service mesh, managing storage operations through the \textit{MictlanX} client. 

The prototype was developed using the Python \textit{FastAPI} framework, chosen for its high performance and its capabilities for the rapid creation of robust APIs. The system's validation was conducted through an image processing use case, demonstrating the feasibility and effectiveness of the \textit{middleware} as a catalyst for integrating complex systems in scientific research environments.  

\textbf{Keywords:} Middleware, Interoperability, Service Mesh, REST API, FastAPI, Nez, Jub, Distributed Systems.

%-----------------------------------------------------------------------------------------------------------------
% PAGINA 7 - INDICE

\clearpage
\addcontentsline{toc}{section}{Índice}
\renewcommand\contentsname{Índice}
\tableofcontents

%-----------------------------------------------------------------------------------------------------------------
% CAPITULOS


\clearpage
\pagenumbering{arabic}
\setcounter{page}{1}
%\input{Capitulo1.tex}

\clearpage
%\input{Capitulo2.tex}

\clearpage
\section{Introducción}

La ciencia de datos moderna depende de la integración de sistemas de software especializados. Este trabajo aborda la creación de un puente de comunicación, un \textit{middleware}, para conectar dos plataformas tecnológicas clave del CINVESTAV, permitiendo así la creación de una malla de servicios unificada para el análisis avanzado de datos.

\subsection{Definición del problema y justificación del proyecto}

La evolución de la ciencia de datos se basa en la integración de un ecosistema tecnológico diverso que incluye \textit{big data}, cómputo en la nube y aprendizaje profundo. Sin embargo, esta especialización a menudo conduce a la creación de ``silos de datos y procesamiento'', donde sistemas de software potentes, diseñados para tareas específicas, operan de forma aislada. Esta falta de comunicación impide la colaboración y limita el potencial sinérgico que podría surgir de su integración, convirtiéndose en un desafío técnico significativo que requiere soluciones para abstraer la complejidad y facilitar un flujo de información cohesivo.

En el Centro de Investigación y de Estudios Avanzados (CINVESTAV) Unidad Tamaulipas, este desafío se manifiesta en dos plataformas clave. Por un lado, \textit{Nez}, un \textit{framework} para el procesamiento de datos a gran escala que implementa el innovador modelo arquitectónico \textit{PuzzleMesh}, validado en dominios como el análisis de tomografías y el procesamiento de imágenes satelitales. Por otro lado, \textit{Jub}, un concentrador y distribuidor de datos especializado en el monitoreo de fenómenos atmosféricos. Un tercer componente, \textit{MictlanX}, gestiona las operaciones de almacenamiento en el ecosistema.  

El problema central es que, a pesar de sus capacidades complementarias, \textit{Nez} y \textit{Jub} carecen de un mecanismo de comunicación nativo y estandarizado. Esta ausencia de interoperabilidad representa una barrera significativa: los usuarios de \textit{Jub} no pueden aprovechar las potentes capacidades de análisis y aprendizaje profundo de \textit{Nez}, y este último no puede ser alimentado de forma automatizada con los flujos de datos gestionados por \textit{Jub}. Este aislamiento tecnológico impide la formación de una malla de servicios de ciencia de datos cohesiva y eficiente.

La justificación de este proyecto reside en la necesidad de romper estos silos. El desarrollo de un \textit{middleware} como puente de comunicación estandarizado permitirá la integración transparente de ambas plataformas, desbloqueando nuevas posibilidades de investigación al combinar el análisis de datos atmosféricos con el procesamiento avanzado de imágenes. Este trabajo es un paso fundamental hacia la creación de una malla de servicios unificada, escalable y eficiente, maximizando el valor de los activos tecnológicos existentes en la institución.

\subsection{Objetivo General}

Diseñar e implementar un \textit{middleware} de acoplamiento ligero que permita la interoperabilidad entre la plataforma \textit{Nez} y el concentrador de servicios \textit{Jub}, habilitando la creación de una malla de servicios de ciencia de datos y aprendizaje profundo.  

\subsection{Objetivos Particulares}

\begin{itemize}
    \item Diseñar la arquitectura del \textit{middleware} para permitir la comunicación eficiente entre las plataformas \textit{Nez} y \textit{Jub}.  
    \item Implementar el \textit{middleware} para lograr un acoplamiento ligero que habilite el intercambio de datos y procesos en tiempo real.  
    \item Integrar servicios de procesamiento distribuido de datos e imágenes mediante algoritmos de aprendizaje profundo.  
    \item Mejorar las interfaces gráficas de usuario existentes para facilitar la gestión y visualización de los datos procesados.  
    \item Elaborar la documentación técnica y los manuales de usuario del sistema para garantizar su correcto uso y mantenimiento futuro.  
\end{itemize}

\subsection{Alcances y limitaciones del Proyecto}

El alcance de este proyecto se centra en la entrega de un prototipo funcional del \textit{middleware} de interoperabilidad. Este prototipo será capaz de recibir solicitudes de \textit{Jub}, orquestar la ejecución de procesos en \textit{Nez} y gestionar el flujo de resultados. El proyecto incluye la validación de la solución a través de un caso de uso específico de procesamiento de imágenes, así como mejoras a la interfaz de usuario de \textit{Jub} para integrar esta nueva funcionalidad.

El proyecto presenta las siguientes limitaciones:

\begin{itemize}
    \item El prototipo se valida con un conjunto limitado de casos de uso, no abarcando la totalidad de las capacidades de \textit{Nez} y \textit{Jub}.  
    \item El manejo de errores se limita a la notificación de fallos, sin implementar mecanismos avanzados de reintentos automáticos o \textit{circuit breaking}.  
    \item La solución no incluye una integración completa con herramientas de monitoreo y observabilidad de nivel de producción como \textit{Prometheus} o \textit{Grafana}.  
    \item El modelo de consulta de estado de los trabajos de procesamiento se basa en sondeo (\textit{polling}) por parte del cliente, en lugar de un sistema de notificaciones proactivas (\textit{webhooks}).  
\end{itemize}

\subsection{Organización del Documento de Tesina}

Este documento se organiza en seis capítulos para presentar de manera clara y estructurada el desarrollo del proyecto. El Capítulo 2 establece el Marco Teórico, donde se definen conceptos fundamentales como mallas de servicios, \textit{middleware} e interoperabilidad, y se describen las tecnologías involucradas, incluyendo el modelo \textit{PuzzleMesh}. El Capítulo 3 se dedica al Diseño Arquitectónico del \textit{middleware}, detallando los requisitos, las decisiones de diseño como la elección de una API REST, y la especificación de la interfaz. El Capítulo 4 describe la Implementación del prototipo, abarcando la selección del \textit{stack} tecnológico, la estructura del código y las estrategias de contenerización con \textit{Docker}. En el Capítulo 5 se presentan las Pruebas y Validación, donde se detalla el escenario de prueba, los resultados de las pruebas de integración y el análisis de rendimiento. Finalmente, el Capítulo 6 expone las Conclusiones y el Trabajo Futuro, resumiendo las contribuciones del proyecto y delineando posibles líneas de investigación y desarrollo.


\clearpage
\section{Marco Teórico}
Este capítulo establece el marco conceptual y tecnológico sobre el cual se fundamenta el proyecto. Se definen los principios de arquitecturas de software modernas, se formaliza el modelo PuzzleMesh, se introduce el rol del middleware como catalizador de la interoperabilidad y se describen las plataformas tecnológicas involucradas.

\subsection{Arquitecturas Orientadas a Microservicios y Mallas de Servicios (Service Mesh)}
Las arquitecturas de microservicios estructuran una aplicación como una colección de servicios pequeños, autónomos y débilmente acoplados. Para gestionar la complejidad en la comunicación entre estos servicios, ha surgido el concepto de Malla de Servicios (Service Mesh). Una malla de servicios es una capa de infraestructura de software dedicada que facilita la comunicación entre microservicios en una arquitectura distribuida. Actúa como un intermediario que gestiona el tráfico de red, proporcionando funcionalidades críticas como el descubrimiento de servicios, el balanceo de carga, la encriptación, la observabilidad y la resiliencia. Su arquitectura se compone de un \textbf{Plano de Datos}, formado por proxies ligeros (sidecars) que interceptan el tráfico, y un \textbf{Plano de Control}, que administra y configura dichos proxies para aplicar políticas de enrutamiento y seguridad.

\subsection{El Modelo Conceptual de PuzzleMesh}
El modelo PuzzleMesh es un marco formal para construir estructuras de procesamiento agnósticas a la infraestructura, utilizando una metáfora de rompecabezas. Este modelo se alinea con los principios de las mallas de servicios al promover la modularidad y la gestión centralizada de componentes reutilizables.  

\subsubsection{La Pieza de Software (P): Unidad Fundamental de Procesamiento}
La unidad básica es la Pieza de Software (P), un artefacto autocontenido que encapsula una aplicación con todos sus componentes necesarios para su despliegue y ejecución. Una pieza funciona como una \enquote{caja negra} con capas definidas para el acceso, interfaces de entrada/salida (E/S), la aplicación, metadatos y dependencias.  

\subsubsection{El Rompecabezas (R): Composición de Flujos de Trabajo}
Un Rompecabezas (R) es una estructura de procesamiento creada al unir un conjunto de Piezas. El orden de ejecución y el flujo de datos se definen mediante un \textbf{Grafo Acíclico Dirigido (DAG)}, donde los nodos son las piezas y los vértices representan las dependencias de datos entre ellas. Un rompecabezas representa un flujo de trabajo completo, como un pipeline de procesamiento de datos.  

\subsubsection{El Metarompecabezas ($\Omega$): Habilitando Flujos de Datos Inter-Sistemas}
PuzzleMesh introduce el concepto de Metarompecabezas ($\Omega$) al encadenar múltiples rompecabezas. Esto permite la creación de flujos de datos complejos que pueden abarcar diferentes departamentos o incluso distintas organizaciones (flujos interinstitucionales).  

\subsubsection{La Malla de Servicios ($\Psi$) en PuzzleMesh}
Finalmente, todas las Piezas, Rompecabezas y Metarompecabezas se incorporan en una Malla de Servicios ($\Psi$). En este contexto, la malla actúa como un repositorio o catálogo centralizado desde el cual las organizaciones pueden seleccionar, componer y reutilizar estos bloques de construcción para crear nuevos servicios.  

\subsection{Middleware como Catalizador de la Interoperabilidad en Sistemas Distribuidos}
El objetivo central de este proyecto es lograr la interoperabilidad entre Nez y Jub, meta que se alcanza a través de un middleware. El middleware es una capa de software que se sitúa entre diferentes aplicaciones, funcionando como un puente o una \enquote{capa de traducción oculta} para facilitar su comunicación y el intercambio de datos. Su función es abstraer la complejidad de la comunicación en un entorno distribuido.

La interoperabilidad es la capacidad de dos o más sistemas de intercambiar información y utilizarla de manera efectiva. Se distinguen varios niveles, desde el \textbf{fundacional} (intercambio básico de datos) hasta el \textbf{semántico} (comprensión compartida del significado de los datos). El middleware es el mecanismo técnico que permite alcanzar la interoperabilidad, proporcionando servicios estandarizados como la transformación de datos y el enrutamiento de mensajes.

\subsection{Plataformas Involucradas}
El ecosistema del proyecto se compone de tres plataformas desarrolladas en el \textbf{CINVESTAV Tamaulipas}:  

\begin{itemize}
    \item \textbf{Nez:} Es el framework para el procesamiento de datos a gran escala que implementa el modelo conceptual de PuzzleMesh. Ha sido utilizado con éxito en dominios como el análisis de imágenes de tomografía y la observación de la Tierra.  
    \item \textbf{Jub:} Es un concentrador y distribuidor de datos diseñado para el monitoreo de fenómenos atmosféricos, actuando como un gestor de datos y punto de acceso para servicios de ciencia de datos.  
    \item \textbf{MictlanX:} Es la plataforma designada para gestionar las operaciones de almacenamiento dentro del ecosistema integrado, manejando la persistencia de los datos que fluyen entre Jub y Nez.  
\end{itemize}

\subsection{Aprendizaje Profundo como Servicio en el Procesamiento de Imágenes y Señales}
La motivación final para integrar estas plataformas es habilitar el uso de algoritmos avanzados de Aprendizaje Profundo (\textit{Deep Learning}). Este subcampo del aprendizaje automático utiliza redes neuronales con múltiples capas para aprender representaciones de datos con altos niveles de abstracción. Esta tecnología es altamente eficaz en el reconocimiento de patrones complejos, especialmente en el procesamiento de imágenes y señales. La integración de Nez y Jub a través del middleware permitirá ofrecer estas capacidades como un servicio (\textit{Deep Learning as a Service -- DLaaS}), donde los usuarios de Jub pueden solicitar análisis complejos que se ejecutarán en la infraestructura de alto rendimiento de Nez.


\clearpage
\section{Sistema Propuesto}

En esta sección se detalla el diseño y la arquitectura del \textit{Nez-Daemon Watcher}, un servicio de software diseñado para funcionar como un **agente de ingesta de datos** para el **Espacio de Almacenamiento Virtual (VSS) de MictlanX**. El rol principal de este componente es actuar como una puerta de enlace robusta y desatendida entre un sistema de archivos local y el ecosistema de almacenamiento distribuido de MictlanX, centrándose en la eficiencia, la resiliencia y la automatización.

\subsection{Análisis de Requerimientos}
El análisis de requerimientos es la base sobre la cual se construye el sistema, definiendo las capacidades y restricciones del mismo.

\subsubsection{Requerimientos Funcionales}
Los requerimientos funcionales describen las operaciones principales que el sistema puede ejecutar para cumplir su misión como agente de ingesta.

\begin{itemize}
    \item \textbf{RF-01: Monitoreo de Directorio.} El sistema debe monitorear de forma continua y recursiva un directorio predefinido (\texttt{watch\_dir}). Esta es la principal fuente de eventos que dispara el flujo de trabajo de carga de archivos.
    
    \item \textbf{RF-02: Carga de Archivos a MictlanX.} Al detectar un nuevo archivo, el sistema debe orquestar su carga al VSS de MictlanX. Este proceso implica almacenar el archivo como un objeto atómico denominado \textbf{Ball} dentro de un contenedor lógico llamado \textbf{Bucket}. La carga debe incluir metadatos clave como el checksum SHA-256 (para integridad), el tamaño, y un factor de replicación (\texttt{replication\_factor}) que instruye a MictlanX sobre cuántas copias del "Ball" deben distribuirse entre los nodos de almacenamiento (\textbf{Peers}) para garantizar la redundancia.
    
    \item \textbf{RF-03: Verificación de Existencia.} Para optimizar el uso de la red y el almacenamiento, el sistema debe verificar si un "Ball" con la misma clave ya existe en el "Bucket" de MictlanX antes de iniciar una carga, evitando así la duplicación de datos.
    
    \item \textbf{RF-04: Gestión de Cuarentena.} Para asegurar la alta disponibilidad, si un archivo falla repetidamente durante el proceso de carga, debe ser movido a un directorio de "cuarentena", aislando los archivos problemáticos sin detener el servicio.
    
    \item \textbf{RF-05: Manejo de Solicitudes de Descarga.} El sistema debe ser capaz de procesar solicitudes de descarga desde MictlanX, actuando como un punto de acceso local al VSS. Los dos mecanismos soportados son a través de archivos \texttt{.mictlanx\_download} y un socket de dominio Unix para comunicación entre procesos (IPC).
\end{itemize}

\subsubsection{Requerimientos No Funcionales}
Los requerimientos no funcionales definen las características de calidad y operativas que garantizan que el sistema sea eficiente y robusto.

\begin{itemize}
    \item \textbf{RNF-01: Concurrencia.} El sistema debe procesar múltiples archivos de forma concurrente mediante un grupo de "trabajadores" asíncronos para maximizar el rendimiento.
    
    \item \textbf{RNF-02: Estabilidad de Archivo.} Antes de procesar un archivo, el sistema debe esperar un breve período para asegurar que la operación de escritura ha finalizado por completo, evitando así la ingesta de datos corruptos o incompletos.
    
    \item \textbf{RNF-03: Resiliencia.} Las operaciones de red deben ser resilientes a fallos transitorios, utilizando una estrategia de reintentos con retroceso exponencial para las llamadas a la API de MictlanX.
    
    \item \textbf{RNF-04: Configurabilidad.} Siguiendo la metodología de 12-Factores, la configuración del sistema (rutas, URI del Router de MictlanX, etc.) debe ser externa al código y gestionada mediante variables de entorno.
    
    \item \textbf{RNF-05: Portabilidad.} El sistema está empaquetado como una imagen Docker \cite{merkel2014docker}, garantizando un entorno de ejecución consistente y simplificando su despliegue.
\end{itemize}

\subsection{El Ecosistema de Almacenamiento MictlanX}
Para comprender el rol y el diseño del \textit{Nez-Daemon Watcher}, es fundamental primero describir el sistema de almacenamiento con el que se integra. MictlanX no es un sistema de almacenamiento monolítico, sino un **Espacio de Almacenamiento Virtual (VSS)**, una abstracción programable sobre un conjunto de recursos de almacenamiento distribuidos. Su propósito es ofrecer una plataforma de almacenamiento flexible, elástica y resiliente.

La arquitectura de MictlanX, basada en el repositorio \texttt{mictlanx-service}, se compone de tres entidades principales que operan de forma desacoplada:
\begin{itemize}
    \item \textbf{Storage Peers (Pares de Almacenamiento):} Son los nodos de almacenamiento individuales y autónomos. Su única responsabilidad es guardar los datos físicos que se les entregan. En la terminología de MictlanX, la unidad atómica de datos es un \textbf{Ball}.
    
    \item \textbf{Storage Replica Management (SPM):} Es el cerebro o la "autoridad de metadatos" del ecosistema. El SPM es un subsistema distribuido que mantiene un registro global del estado del clúster. Sabe qué Peers están activos, qué "Balls" existen, en qué "Peers" reside cada réplica de un "Ball", y cómo están organizados lógicamente dentro de contenedores llamados \textbf{Buckets}.
    
    \item \textbf{Router (Enrutador):} Es el único punto de entrada y el orquestador de operaciones de entrada/salida (I/O) para todo el VSS. Los clientes, como el \textit{Nez-Daemon Watcher}, no interactúan directamente con los Peers o el SPM. Toda la comunicación se realiza a través de la API del Router.
\end{itemize}

Cuando el Watcher necesita cargar un archivo, se comunica con el Router. El Router, a su vez, consulta al SPM para tomar decisiones inteligentes sobre dónde y cómo almacenar el archivo (por ejemplo, seleccionando los Peers menos cargados y asegurando el factor de replicación solicitado). Una vez tomada la decisión, el Router guía al cliente para que complete la transferencia de datos. Esta arquitectura desacoplada, que separa la gestión de metadatos (el "plano de control" del SPM) de las operaciones de datos (el "plano de datos" del Router y los Peers), es lo que le da a MictlanX su escalabilidad y flexibilidad. El \textit{Nez-Daemon Watcher} actúa, por tanto, como un cliente especializado que traduce eventos del sistema de archivos en operaciones complejas dentro de este ecosistema distribuido.

\subsection{Arquitectura del Sistema}
La arquitectura del \textit{Nez-Daemon Watcher}, descrita en esta sección, está diseñada específicamente para funcionar como un componente de borde (edge component) eficiente y robusto para el VSS de MictlanX. Su diseño modular y orientado a eventos le permite abstraer la complejidad del ecosistema de almacenamiento y operar de forma autónoma. La Figura \ref{fig:arquitectura_watcher} ilustra la interacción entre sus componentes internos.

\begin{figure}[h]
    \centering
    % Código fuente en Diagramas UML/arquitectura.puml
    \includegraphics[width=\textwidth]{Diagramas UML/arquitectura.png}
    \caption{Diagrama de alto nivel de la arquitectura del Watcher.}
    \label{fig:arquitectura_watcher}
\end{figure}

\subsubsection{Componentes del Watcher}
\begin{itemize}
    \item \textbf{Orquestador Principal (\texttt{watcher.py}):} Inicia y supervisa todos los componentes. Realiza una comprobación de salud inicial contra el Router de MictlanX antes de empezar a operar.
    
    \item \textbf{Observador del Sistema de Archivos (\texttt{NewFileHandler}):} Utiliza \texttt{watchdog} \cite{watchdog2023docs} para detectar eventos y actúa como un puente hacia el bucle de eventos asíncrono de la aplicación.
    
    \item \textbf{Cola de Tareas (\texttt{asyncio.Queue}):} Búfer central que desacopla la detección de eventos del procesamiento, permitiendo al sistema gestionar picos de trabajo.
    
    \item \textbf{Trabajadores (\texttt{worker}):} Tareas asíncronas que consumen eventos de la cola y ejecutan la lógica de negocio.
    
    \item \textbf{Módulo de Operaciones MictlanX (\texttt{mictlanx\_ops.py}):} Abstrae la comunicación con el Router de MictlanX. Su función más importante es implementar el proceso de carga en dos fases dictado por la arquitectura de MictlanX.
    
    \item \textbf{Servidor de Socket (\texttt{socket\_server.py}):} Expone un socket de dominio Unix para recibir órdenes de otros procesos locales.
\end{itemize}

\subsection{Diagrama de Contexto}
El diagrama de contexto, mostrado en la Figura \ref{fig:contexto}, posiciona al sistema como un servicio intermediario entre el entorno local y el VSS de MictlanX.

\begin{figure}[h]
    \centering
    % Código fuente en Diagramas UML/contexto.puml
    \includegraphics[width=\textwidth]{Diagramas UML/contexto.png}
    \caption{Diagrama de Contexto del Sistema Watcher.}
    \label{fig:contexto}
\end{figure}

Las entidades externas son:
\begin{itemize}
    \item \textbf{Usuario/Administrador:} Inicia y configura el servicio.
    \item \textbf{Sistema de Archivos:} El entorno local que el watcher monitorea.
    \item \textbf{Router de MictlanX (VSS):} El punto de entrada al sistema de almacenamiento distribuido, con el que se comunica vía API REST.
    \item \textbf{Aplicación Cliente Local:} Proceso local que solicita descargas vía socket.
\end{itemize}

\subsection{Diagramas de Casos de Uso}
Los casos de uso definen las interacciones funcionales clave. La Figura \ref{fig:casos_uso} muestra su representación gráfica.

\begin{figure}[h]
    \centering
    % Código fuente en Diagramas UML/casos_de_uso.puml
    \includegraphics[width=0.8\textwidth]{Diagramas UML/casos_de_uso.png}
    \caption{Diagrama de Casos de Uso.}
    \label{fig:casos_uso}
\end{figure}

\subsection{Diagramas UML}
El diagrama de secuencia en la Figura \ref{fig:diagrama_secuencia} detalla el flujo de la carga de un archivo, evidenciando la comunicación exclusiva con el Router de MictlanX.

\begin{figure}[h]
    \centering
    % Código fuente en Diagramas UML/secuencia_carga.puml
    \includegraphics[width=\textwidth]{Diagramas UML/secuencia_carga.png}
    \caption{Diagrama de Secuencia para la carga de un nuevo archivo.}
    \label{fig:diagrama_secuencia}
\end{figure}

La secuencia es:
\begin{enumerate}
    \item El \texttt{Observador} detecta un archivo y encola una tarea.
    \item Un \texttt{Trabajador} toma la tarea.
    \item El \texttt{Trabajador} invoca al módulo de \texttt{Operaciones MictlanX}.
    \item El módulo de operaciones envía una petición de registro de metadatos al \texttt{Router de MictlanX}. El Router consulta al SPM y devuelve un ID de tarea.
    \item El módulo de operaciones envía el contenido del archivo al \texttt{Router}, usando el ID de tarea. El Router gestiona la transferencia a los Peers apropiados.
\end{enumerate}

\subsection{Diseño de Entradas y Salidas}
Las entradas y salidas del sistema se definen a nivel de datos y protocolos.

\subsubsection{Entradas del Sistema}
\begin{itemize}
    \item \textbf{Archivos en Directorio:} Archivos binarios de cualquier tipo.
    \item \textbf{Mensajes de Socket:} Cadenas de texto UTF-8 con la ruta del recurso a descargar.
    \item \textbf{Variables de Entorno:} Pares clave-valor para la configuración.
\end{itemize}

\subsubsection{Salidas del Sistema}
\begin{itemize}
    \item \textbf{Llamadas a la API del Router de MictlanX:} Peticiones HTTP POST a los endpoints \texttt{/api/v4/buckets/metadata} y \texttt{/api/v4/buckets/data/\{task\_id\}}.
    \item \textbf{Archivos en Cuarentena:} Archivos movidos al directorio de cuarentena.
    \item \textbf{Registros de Actividad (Logs):} Salida de texto estructurada en la consola.
    \item \textbf{Archivos Descargados:} Archivos escritos en el disco local.
\end{itemize}

\subsection{Algoritmos Principales}
A continuación se describen los algoritmos que definen el flujo de control y la lógica de negocio del Watcher.

\subsubsection{Algoritmo del Trabajador Principal}
Este algoritmo representa el bucle de eventos de cada trabajador, responsable de consumir tareas y despacharlas a la lógica apropiada.

\begin{algorithm}
\caption{Procesamiento de Tareas del Trabajador}
\begin{algorithmic}[1]
\While{el sistema está activo}
    \State \texttt{tarea} $\leftarrow$ esperar y obtener de \texttt{ColaDeTareas}
    \If{\texttt{tarea} es un evento de archivo}
        \State \texttt{ruta\_archivo} $\leftarrow$ obtener ruta de \texttt{tarea}
        \If{\texttt{ruta\_archivo} es una solicitud de descarga}
            \State \Call{procesarSolicitudDeDescarga}{\texttt{ruta\_archivo}}
        \Else
            \State \Call{esperarEstabilidadDeArchivo}{\texttt{ruta\_archivo}}
            \If{\Call{archivoYaExisteEnMictlanX}{\texttt{ruta\_archivo}}}
                \State registrar "Archivo ya existe, omitiendo."
                \State \textbf{continue}
            \EndIf
            \If{\Call{subirArchivoAMictlanX}{\texttt{ruta\_archivo}} es exitoso}
                 \State registrar "Archivo subido con éxito."
            \Else
                 \State registrar "Fallo en la carga del archivo."
                 \State \Call{moverArchivoACuarentena}{\texttt{ruta\_archivo}}
            \EndIf
        \EndIf
    \ElsIf{\texttt{tarea} es una solicitud de socket}
        \State \texttt{ruta\_descarga} $\leftarrow$ obtener ruta de \texttt{tarea}
        \State \Call{procesarDescargaDesdeSocket}{\texttt{ruta\_descarga}}
    \EndIf
\EndWhile
\end{algorithmic}
\end{algorithm}

\subsubsection{Algoritmo de Carga de Archivo a MictlanX}
Este algoritmo es crucial, pues su diseño de dos pasos es una consecuencia directa de la arquitectura desacoplada de MictlanX. La separación del registro de metadatos y la carga de datos permite al Router coordinar la replicación y preparar a los Peers antes de recibir el contenido, aumentando la escalabilidad y robustez del VSS.

\begin{algorithm}
\caption{Carga de Archivo a MictlanX}
\begin{algorithmic}[1]
\Procedure{subirArchivoAMictlanX}{\texttt{ruta\_archivo}}
    \State \texttt{checksum} $\leftarrow$ \Call{calcularSHA256}{\texttt{ruta\_archivo}}
    \State \texttt{tamaño} $\leftarrow$ \Call{obtenerTamaño}{\texttt{ruta\_archivo}}
    \State \texttt{clave} $\leftarrow$ \Call{sanitizarRutaRelativa}{\texttt{ruta\_archivo}}
    
    \State \texttt{metadatos} $\leftarrow$ crear estructura con \texttt{bucket\_id}, \texttt{clave}, \texttt{checksum}, \texttt{tamaño}, \texttt{replication\_factor}
    
    \State \textit{// Paso 1: Registrar metadatos en el Router}
    \State \texttt{respuesta\_meta} $\leftarrow$ \Call{POST}{"/api/v4/buckets/metadata", \texttt{metadatos}}
    \If{\texttt{respuesta\_meta} es un error}
        \State \Return \texttt{falso}
    \EndIf
    \State \texttt{id\_tarea} $\leftarrow$ \Call{obtenerID}{\texttt{respuesta\_meta}}
    
    \State \textit{// Paso 2: Subir contenido de datos al Router usando el ID de tarea}
    \State \texttt{respuesta\_datos} $\leftarrow$ \Call{POST}{"/api/v4/buckets/data/\texttt{id\_tarea}", \texttt{ruta\_archivo}}
    \If{\texttt{respuesta\_datos} es un error}
        \State \Return \texttt{falso}
    \EndIf
    
    \State \Return \texttt{verdadero}
\EndProcedure
\end{algorithmic}
\end{algorithm}



\clearpage
\input{Capitulo6.tex}

\clearpage
\addcontentsline{toc}{section}{Índice de figuras}
\renewcommand\listfigurename{Índice de figuras}

\listoffigures

\clearpage
\addcontentsline{toc}{section}{Índice de cuadros}
\renewcommand\listtablename{Índice de cuadros}
\listoftables

\clearpage
\addcontentsline{toc}{section}{Índice de algoritmos}
\renewcommand\listalgorithmname{Índice de algoritmos}
\listofalgorithms

%-----------------------------------------------------------------------------------------------------------------
% REFERENCIAS


\clearpage
%Let's cite! The Einstein's journal paper \cite{dirac} and the Dirac's 
%book \cite{einstein} are physics related items. 

%\Urlmuskip=0mu plus 1mu\relax
\addcontentsline{toc}{section}{Referencias} 
\printbibliography
 
\end{document}
